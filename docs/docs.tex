%! suppress = TooLargeSection
%! suppress = SentenceEndWithCapital
%! suppress = TooLargeSection
% Preamble
\documentclass[11pt]{PyRollDocs}
\usepackage{textcomp}
\usepackage{csquotes}
\usepackage{wasysym}

\addbibresource{refs.bib}
% Document
\begin{document}

    \title{Karman Power and Labour PyRolL Plugin}
    \author{Christoph Renzing}
    \date{\today}

    \maketitle

    The PyRolL plugin pyroll-neutral-line-estimator provides different estimators for the neutral point which markes the point of zero shear stresses inside the roll gap.


    \section{Model approach}\label{sec:model-approach}

    For rolling in grooves as for flat rolling, the neutral point is the point inside the roll gap where the shear stresses become zero.
    Hence, rolling is a three-dimensional forming process, the neutral point isn't a point but a curved plane inside the roll gap.
    For flat rolling as for groove rolling, the actual plane is often assumed as a point marking the horizontal coordinate inside the roll gap.
    In the case of groove rolling, \textcite{} stated that due to the spreading of the material the roll gap has to be divided into a forward-slip area, a backward-slip area
    as well as two areas which are called spreading areas.
    \autoref{} shows a simplified sketch of the areas during rolling.
    
    The pyroll-neutral-point-estimator plugin provides different simplified solutions derived by different authors for flat rolling.
    Featured model equations are taken from \textcite{}, \textcite{}, \textcite{} as well as \textcite{}.

    \paragraph{Ford-Ellis-Bland Solution}\\

    \paragraph{Osborn Solution}\\

    \paragraph{Sims Solution}\\

    \paragraph{Siebel Solution}\\

    \section{Usage instructions}\label{sec:usage-instructions}
    The plugin can be loaded under the name \texttt{pyroll\_karman\_power\_and\_labour}.

    An implementation of the \lstinline{roll_force} and \lstinline{mean_neutral_plane_position} hook on \lstinline{RollPass} is provided.
    Furthermore, an implementation of the \lstinline{roll_torque} hook on \lstinline{RollPass.Roll} is provided.

    Additionally, hooks on \lstinline{RollPass} are defined, which are used in the calculation, as listed in \autoref{tab:hookspecs}.
    The hooks \lstinline{mean_front_tension}, \lstinline{mean_back_tension} and \lstinline{coulomb_friction_coefficient} have to set and adjusted individually.

    \begin{table}
        \centering
        \caption{Hooks specified by this plugin.}
        \label{tab:hookspecs}
        \begin{tabular}{ll}
            \toprule
            Hook name                               & Meaning                                                \\
            \midrule
            \texttt{coulomb\_friction\_coefficient} & Coulomb's friction coefficient $\mu$                   \\
            \texttt{mean\_back\_tension}            & Mean back tension of the roll pass $\sigma_{x,0}$      \\
            \texttt{mean\_front\_tension}           & Mean front tension of the roll pass $\sigma_{x,1}$     \\
            \texttt{mean\_neutral\_plane\_position} & Mean neutral plane postion $x_N$                       \\
            \texttt{karman\_solution}               & KarmanSolver object with solution values as attributes \\
            \bottomrule
        \end{tabular}
    \end{table}

    \printbibliography


\end{document}